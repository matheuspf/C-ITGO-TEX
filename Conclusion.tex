\section{Conclusion} \label{sec:Conclusion}

This work presents a modified iterative topographical global optimization algorithm, denominated C-ITGO, specialized in solving constrained optimization problems. The method developed in this paper uses a custom topographical heuristic, based on a stochastic application of the three steps criteria comparison, along with a space reduction procedure. In addition to being very simple conceptually, the C-ITGO algorithm is very generic in the sense that any local search procedure can be used to tackle a specific problem, being it continuous or discrete.

The C-ITGO method was compared against several algorithms found in literature, some presenting state-of-the-art results, in eight difficult constrained engineering optimization problems. In all tests performed, the C-ITGO outperforms any competing method regarding the mean number of function evaluations (MNFEs) to converge to the best-known solution found, with performance gain in some problems being of an order of magnitude over the best performing methods compared.

We also presented several statistical tests comparing C-ITGO against the other methods used in this work. In this work, was shown that C-ITGO outperforms the other competing methods, presenting a better performance that is significantly different at 0.05 level with respect to the MNFEs required to converge to optimal or near-optimal solutions for all the engineering design problems compared.

As future work, we suggest the following possibilities:


\begin{itemize}

    \item use new, possibly even more complex optimization problems to test the performance of C-ITGO.

    \item propose an adaptive method to auto adjust the parameters that can influence on C-ITGO performance.

    \item develop a parallel version aiming at faster execution time, even if this is not the main measure of performance.


\end{itemize}