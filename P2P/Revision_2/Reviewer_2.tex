% !TEX root = ../../main.tex

\section*{Reviewer \#2}

\begin{revAns}{Although the authors made a significant improvement in the revised paper, there remain some important questions that I would ask the authors to consider. They mainly concern a newly added section (Appendix) where results on test classes are reported.

1) Please indicate the number of evaluations executed in local searches. Are these numbers currently included in the numbers of Table 18?}

Yes, the number of function evaluations executed in the local searches are included in the calculations of the mean number of function evaluations. 

Actually, the local searches are responsible for the most part of the function calls, as stated in page 17, paragraph 3, where we comment about the impact of the maximum number of function evaluations allowed for each call of the local search procedure.

\end{revAns}


\begin{revAns}{2) When several runs are executed by a stochastic method for solving a particular problem, some of them can be unsuccessful and for them maximal allowed number of evaluations can be reached. If these runs are excluded from calculating the averages, the results can be falsely advantageous for the considered method with respect to the other compared algorithms. Please comment upon this aspect and indicate the percentage of unsolved problems (as done for example in doi 10.1016/j.amc.2017.05.014 - Metaheuristic vs. deterministic global optimization algorithms: The univariate case). It is also a good practice to indicate other statistics (for example, standard deviations) for the results given by stochastic methods.}

When considering the engineering design problems, all solutions found by C-ITGO were completely feasible, respecting all imposed constraints (see page 23, paragraph 3). Also, the execution of C-ITGO is halted only when the best-found solution in a run is considerably close to the global optimum, as described in Table 2, page 23. The standard deviation achieved by C-ITGO is very small in all problems, with the fitness of the worst solution being very close to the fitness of the best solution. In fact, it is one of the methods with smallest standard deviation in all cases when compared to the other methods considered. So, we believe we can consider that all runs of the C-ITGO method were successful for all the eight engineering design problems.

When it comes to the GKLS class of problems, a run is considered successful if it satisfies the criteria described in \cite{ADC2}, which was also emphasized in the appendix of the text, more specifically in page 52, equation 3. If a method executes $10^6$ function evaluations and does not achieve convergence, its execution is halted and that run is marked as unsuccessful (as displayed in Table 18 in the same fashion as \cite{NAT}).

We used the same approach as described in \cite{NAT} to apply C-ITGO to the GKLS class of problems, running the algorithm 100 times for each of the 100 problems of each class (with 8 classes, this means 800 different problems, with a total of 80,000 runs), as described in page 52. We then average the number of function/gradient calls for each problem, that is, we take the mean of the 100 runs for each of the 800 problems. Every run of C-ITGO (every one of the 80,000 runs) was successful and achieved convergence under $10^6$ function calls.

Now we also included in the text the minimum, maximum and the standard deviation for all the 10,000 runs for each class of problems and also after averaging all the 100 runs of the same problem (Table 19, page 54). We again emphasize that none of the 80,000 runs (which are all successful) exceeded $10^6$ function/gradient calls. We then conclude that every run of C-ITGO for all the 800 problems of the GKLS class of problems was successful, with none of them exceeding the maximum number of function evaluations allowed ($10^6$).

%All the logs can be found at the same link of the source code, which, if compiled and ran, would give the exact same results as presented in the text. W

\end{revAns}


\begin{revAns}{Please do not use term unconstrained optimization for box-constrained problems to avoid misunderstanding (unconstrained optimization problems have no any constraints while the problems in Appendix do have them).}

We agree with the reviewer. Using the term unconstrained optimization for bound constrained optimization is misleading and wrong. The current version of the paper was modified accordingly. %We used the expression `unconstrained optimization' to denote bound constrained optimization for simplicity, which is not accurate.

\end{revAns}


\begin{revAns}{Highlights: The developed method can solve real, integer and mixed-integer constrained problems -$>$ but only continuous problems are considered.}

Integer and mixed-integer constrained problems were also considered. The Gear Train (GT) and the Multiple Disk clutch brake (MD) are examples of integer constrained optimization problems, while the Speed Reducer (SRI and SRII) and Pressure Vessel (PV) are mixed-integer constrained optimization problems.

In section 3, \textbf{Computational Results}, we comment about each problem, explicitly stating in both the text and in the mathematical formulation whether it is a continuous, integer, or mixed-integer constrained problem.

\end{revAns}



\begin{revAns}{Finally, bibliography is needed to be polished: please check the references Gaviano et al. (2003), Paulavcius and Zilinskas (2014), Paulavcius and Zilinskas (2016), Sergeyev and Kvasov (2006), Sergeyev et al. (2013), Sergeyev et al. (2018), that are either incorrect or contain errors.}

This was already solved and the references were double checked. They are corrected now on the current version of the paper.

\end{revAns}