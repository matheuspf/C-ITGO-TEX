\begin{frontmatter}

%\title{A Constrained Iterative Topographical Global \\ Optimization Applied to Engineering Optimization}
\title{A Constrained ITGO Heuristic \\ Applied to Engineering Optimization}

%\tnotetext[mytitlenote]{Fully documented templates are available in the elsarticle package on \href{http://www.ctan.org/tex-archive/macros/latex/contrib/elsarticle}{CTAN}.}


%% Group authors per affiliation:
%\author{Elsevier\fnref{footUFT}}
%\address{Radarweg 29, Amsterdam}
%\fntext[footUFT]{Universidade Federal do Tocantins}


\author[label1]{Matheus~Pedroza~Ferreira}
\ead{matheuspedrozaferreira@uft.edu.br}

\author[label1]{Marcelo~Lisboa~Rocha\corref{cor1}}
\ead{mlisboa@uft.edu.br}

\author[label2]{Ant\^onio~J.~Silva~Neto}
\ead{ajsneto@iprj.uerj.br}

\author[label3]{Wagner~F.~Sacco}
\ead{wagner.sacco@ufopa.edu.br}



\address[label1]{Departamento de Ci\^encia da Computa\c{c}\~ao, Universidade Federal do Tocantins, Quadra 109 Norte, Avenida NS-15, ALCNO-14, Palmas, Tocantins, Brazil}

\address[label2]{Departamento de Engenharia Mec\^anica e Energia, Instituto Polit\'ecnico, Universidade do Estado do Rio de Janeiro, IPRJ/UERJ, RJ, Brazil}

\address[label3]{Instituto de Engenharia e Geociências, Universidade Federal do Oeste do Pará, PA, Brazil}



\cortext[cor1]{Corresponding authors}










\iffalse

\fntext[footMat]{Dept. de Ci\^encia da Computa\c{c}\~ao, Universidade Federal do Tocantins. \\ \textit{E-mail}: \textbf{matheuspedrozaferreira@uft.edu.br}}

\fntext[footLis]{Dept. de Ci\^encia da Computa\c{c}\~ao, Universidade Federal do Tocantins. \\ \textit{E-mail}: \textbf{mlisboa@uft.edu.br}}

\fntext[footAnt]{Departamento de Engenharia Mec\^anica e Energia, Instituto Politécnico, Universidade do Estado do Rio de Janeiro IPRJ/UERJ. \\ \textit{E-mail}: \textbf{ajsneto@iprj.uerj.br}}

\fi





%% or include affiliations in footnotes:
%\author[mymainaddress,mysecondaryaddress]{Elsevier Inc}
%\ead[url]{www.elsevier.com}

%\author[mysecondaryaddress]{Global Customer Service\corref{mycorrespondingauthor}}
%\cortext[mycorrespondingauthor]{Corresponding author}
%\ead{support@elsevier.com}







% P2P!!!!!!!!

\renewcommand{\abstractname}{Point-to-point changes}

\begin{abstract}
   %\section*{Point-to-point changes}

% !TEX root = ../../main.tex

\section*{Reviewer \#1}

\begin{revAns}{Some important points in the experimental part of the revised paper still however persist. They regard the adopted comparison strategy between deterministic and metaheuristic algorithms. When I said "(please take into account that the number of function evaluations is to be multiplied by the number of runs of a metaheuristic method - 25 in this paper)" this did not mean to use a particular "type of comparison" as authors replied. This just means that to obtain the reported numbers of function evaluations the authors needed to perform 25 (or 100 in new experiments) launches of their method for each test function and then to average the results. Thus, while the competitors used, let us say, a thousand evaluations, the authors claimed to use, in average, 500 evaluations (which is apparently better!) but, in fact, they used, in average, 25*500 (or 100*500) evaluations of a test function (which is much worse!). Thus, if we have a practical problem to solve, the competitors solve this problem in 1000 evaluations, while the authors' method can solve it in 10 evaluations in a particular run, in 500 evaluations in another run, or probably does not solve the problem at all. To have a higher possibility to solve the problem, a higher number of launches of the method is needed or a careful tuning of the multiple parameters of the method has to be done thus increasing the required computational costs.
    
Therefore, I still do not find the reported numerical results to be fair. As I could understand, the methodology reported in Sergeyev et al. (2018) aimed at considering all these worst, best, and average cases in a unique diagram. The authors instead have just taken results from this paper and mixed them with their own ones thus producing superficial and misleading conclusions. Maybe I am wrong but neither from the revised paper nor from the authors' brief reply I could see the opposite conclusion.}

The methodology used to achieve the results presented in the paper may be not very clear, so we now make a thorough discussion about the methods used.


\subsubsection*{Engineering Design Problems}

We consider eight engineering design problems, consisting of continuous, integer and mixed-integer constrained optimization problems. We compared the results achieved by C-ITGO against 19 competitive meta-heuristic methods, with some of them reporting state of the art results for the problems considered.

For each of the eight engineering design problems, we run the C-ITGO algorithm 25 times, saving all the statistics of each run. The choice of the number of runs for each problem is completely arbitrary. We chose the number 25 to match the results of one of the most competitive methods considered, the IAPSO \citep{IAPSO}. However, as the only statistic considered is the Mean Number of Function Evaluations (MNFEs), we can safely compare C-ITGO against the other meta-heuristics, even if they execute a different number of runs for each problem.

For each of the 25 runs for a problem, we restarted the random number generator with a different seed, so as to couple with the stochastic nature of C-ITGO. If the same seed is used for each run, the exactly same results are achieved, which is obviously not wanted.

However, the same sequence of seeds was used for all problems. Specifically, we started with the seed number $270001$ (this is an arbitrary value used by some authors \citep{BRKGABB}) for problem 1 and increased the seed value by 1 for each run. That is, for run 1, we used seed number 270001, for run 2, the seed number is 270002, and so on.

Any other sequence of seeds could be used. In fact, we could choose a particular sequence of seeds for each problem that would give the best results on average. This, of course, would bias the results. This was not the case with C-ITGO, as we used the same arbitrary sequence of seeds for all the eight problems.

The statistics current reported for each problem are: the function value of the best solution found among the 25 runs (\textbf{Best}), the mean function value of the 25 solutions found (\textbf{Mean}), the function value of the worst solution found among the 25 runs (\textbf{Worst}), the standard deviation of the function value of the 25 solutions (\textbf{SD}) and the mean number of function evaluations needed to achieve convergence (\textbf{MNFEs}), also calculated among the 25 runs.

%We now included in the main text a table for each problem reporting more statistics regarding the number of function evaluations. The minimum, first quartile, mean, third quartile, maximum and standard deviation of the number of function evaluations required to achieve convergence are presented.

The convergence criteria are discussed in the text and are specific to each problem (see page 23). We chose to adopt these criteria because the problems being solved present very different characteristics, like the results presented by the competing methods. Still, the C-ITGO algorithm achieves very small standard deviation for the function value in all problems considered (smaller than most competing methods for all problems).

We also note, again, that all solutions found by C-ITGO for all the engineering problems are completely feasible (they respect all imposed constraints on all runs), as stated in page 23, paragraph 3. So, we can consider that every run of the C-ITGO method is successful, having no failed runs given the convergence criteria.

Considering the methodology used, the authors consider that the comparison of the C-ITGO method against the competing meta-heuristics is completely clear, valid and fair. Of course, all the methodology presented here can be found in the main text.

%Also, the complete log of the runs for all the eight engineering design problems can be found in the link to the source code, at the end of the point-to-point review.


\subsubsection*{GKLS}

For the GKLS class of problems, we used exactly the same methodology used in \cite{NAT} for meta-heuristics. We considered 100 different problems for each of the eight classes, ranging from 2 to 5 dimensions, each consisting of an easy and a hard case. For each problem, C-ITGO was executed 100 times, saving the statistic for all runs.

The first thing to consider is that every single run of C-ITGO was successful. That is, none of the 80,000 runs (8 classes, 100 problems, 100 runs for each problem) required more than $10^6$ function evaluations to achieve convergence. That is, not even a single run has failed.

For all classes, we fed the random number generator with the seed value 270001 for problem 1, 270002 for problem 2, and so on, up to 270100 for problem 100. For each problem, we save the number of function evaluations required to achieve convergence.

Table 18 in the main text shows the mean number of function evaluations required to achieve convergence for all the eight classes. This table presents the same results as the Table 1 of \cite{NAT}, when comparing deterministic and meta-heuristics.

According to the reviewer's opinion, the results presented in the main text are `misleading' and `superficial', even considering the fact that the results presented were achieved in the exactly same way as described in \cite{NAT}. So, the authors decided to include some more data for the comparison of the developed method.

We added to the main text, in the appendix section, plots showing the \textit{Operational Zones} of the C-ITGO method, together with the \textit{Operational Characteristics} of the competing deterministic methods, for the easy and hard cases of the 5-dimensional class. 

Also, a new table with some more statistics regarding the number of function evaluations was added. We included the minimum, maximum and the standard deviation for all the 10,000 runs for each class of problems and also after averaging all the 100 runs of the same problem (Table 19, page 54).

%Again, all the logs of all runs are available in the link of the source code, along with the plots, tables, and images.

We hope that all those additions to the text convince the reviewer that the methods and comparison that were used in this work are completely fair.



\end{revAns}
% !TEX root = ../../main.tex

\section*{Reviewer \#2}

\begin{revAns}{Although the authors made a significant improvement in the revised paper, there remain some important questions that I would ask the authors to consider. They mainly concern a newly added section (Appendix) where results on test classes are reported.

1) Please indicate the number of evaluations executed in local searches. Are these numbers currently included in the numbers of Table 18?}

Yes, the number of function evaluations executed in the local searches are included in the calculations of the mean number of function evaluations. 

Actually, the local searches are responsible for the most part of the function calls, as stated in page 17, paragraph 3, where we comment about the impact of the maximum number of function evaluations allowed for each call of the local search procedure.

\end{revAns}


\begin{revAns}{2) When several runs are executed by a stochastic method for solving a particular problem, some of them can be unsuccessful and for them maximal allowed number of evaluations can be reached. If these runs are excluded from calculating the averages, the results can be falsely advantageous for the considered method with respect to the other compared algorithms. Please comment upon this aspect and indicate the percentage of unsolved problems (as done for example in doi 10.1016/j.amc.2017.05.014 - Metaheuristic vs. deterministic global optimization algorithms: The univariate case). It is also a good practice to indicate other statistics (for example, standard deviations) for the results given by stochastic methods.}

When considering the engineering design problems, all solutions found by C-ITGO were completely feasible, respecting all imposed constraints (see page 23, paragraph 3). Also, the execution of C-ITGO is halted only when the best-found solution in a run is considerably close to the global optimum, as described in Table 2, page 23. The standard deviation achieved by C-ITGO is very small in all problems, with the fitness of the worst solution being very close to the fitness of the best solution. In fact, it is one of the methods with smallest standard deviation in all cases when compared to the other methods considered. So, we believe we can consider that all runs of the C-ITGO method were successful for all the eight engineering design problems.

When it comes to the GKLS class of problems, a run is considered successful if it satisfies the criteria described in \cite{ADC2}, which was also emphasized in the appendix of the text, more specifically in page 52, equation 3. If a method executes $10^6$ function evaluations and does not achieve convergence, its execution is halted and that run is marked as unsuccessful (as displayed in Table 18 in the same fashion as \cite{NAT}).

We used the same approach as described in \cite{NAT} to apply C-ITGO to the GKLS class of problems, running the algorithm 100 times for each of the 100 problems of each class (with 8 classes, this means 800 different problems, with a total of 80,000 runs), as described in page 52. We then average the number of function/gradient calls for each problem, that is, we take the mean of the 100 runs for each of the 800 problems. Every run of C-ITGO (every one of the 80,000 runs) was successful and achieved convergence under $10^6$ function calls.

Now we also included in the text the minimum, maximum and the standard deviation for all the 10,000 runs for each class of problems and also after averaging all the 100 runs of the same problem (Table 19, page 54). We again emphasize that none of the 80,000 runs (which are all successful) exceeded $10^6$ function/gradient calls. We then conclude that every run of C-ITGO for all the 800 problems of the GKLS class of problems was successful, with none of them exceeding the maximum number of function evaluations allowed ($10^6$).

%All the logs can be found at the same link of the source code, which, if compiled and ran, would give the exact same results as presented in the text. W

\end{revAns}


\begin{revAns}{Please do not use term unconstrained optimization for box-constrained problems to avoid misunderstanding (unconstrained optimization problems have no any constraints while the problems in Appendix do have them).}

We agree with the reviewer. Using the term unconstrained optimization for bound constrained optimization is misleading and wrong. The current version of the paper was modified accordingly. %We used the expression `unconstrained optimization' to denote bound constrained optimization for simplicity, which is not accurate.

\end{revAns}


\begin{revAns}{Highlights: The developed method can solve real, integer and mixed-integer constrained problems -$>$ but only continuous problems are considered.}

Integer and mixed-integer constrained problems were also considered. The Gear Train (GT) and the Multiple Disk clutch brake (MD) are examples of integer constrained optimization problems, while the Speed Reducer (SRI and SRII) and Pressure Vessel (PV) are mixed-integer constrained optimization problems.

In section 3, \textbf{Computational Results}, we comment about each problem, explicitly stating in both the text and in the mathematical formulation whether it is a continuous, integer, or mixed-integer constrained problem.

\end{revAns}



\begin{revAns}{Finally, bibliography is needed to be polished: please check the references Gaviano et al. (2003), Paulavcius and Zilinskas (2014), Paulavcius and Zilinskas (2016), Sergeyev and Kvasov (2006), Sergeyev et al. (2013), Sergeyev et al. (2018), that are either incorrect or contain errors.}

This was already solved and the references were double checked. They are corrected now on the current version of the paper.

\end{revAns}
%% !TEX root = ../main.tex

\section*{Observations}

One thing that is worthwhile to explain is how we calculated the Mean Number of Function Evaluations (MNFEs) in this paper. We used the same formula as the criteria C3 as described in \cite{ADC}. Given a number $N$ of problems, a number $T$ of runs for a given problem, and the total number of function evaluations $fes_{i, j}$ required to reach a certain convergence criteria for the problem $i$ on the run number $j$, the $MNFEs$ is given by:

$$MNFEs = \frac{1}{n.t}\sum_{n=1}^N\sum_{t=1}^Tfes_{i, j}$$\\

For all the engineering problems, $N=1$ and $T=25$, while for each class of the GKLS problems (for all the 8 classes with 100 problems each), $N=100$ and $T=100$.


\vspace{0.5cm}
\end{abstract}

\renewcommand{\abstractname}{Abstract}








\begin{abstract}
    Nonlinear optimization is an active line of research, given the wide range of scientific fields that benefit from its development. In the last years, the meta-heuristics proved to be one of the most effective methods to tackle difficult optimization problems, providing an alternative in cases where exact methods would be unfeasible. In this work, we present a method based on the Iterative Topographical Global Optimization meta-heuristic, which we call C-ITGO, incorporating specific mechanisms to solve nonlinearly constrained optimization problems. We use the method developed in this work to optimize eight complex engineering design problems and compare the results obtained here against those obtained with several other methods found in the literature. In the tests performed, C-ITGO outperforms all competing methods, achieving state of the art results for the problems considered.
\end{abstract}


\begin{keyword}
Meta-heuristics \sep ITGO \sep Optimization \sep Engineering Problems.
\end{keyword}


%\begin{keyword}
%Metaheuristics \sep BRKGA \sep Global Optimization \sep Local Search.
%\MSC[2010] 00-01\sep  99-00
%\end{keyword}


\end{frontmatter}


%\blfootnote{\fnt{9} \textit{Key words.}  Metaheuristics, BRKGA, Global Optimization, Local Search.}




%\linenumbers