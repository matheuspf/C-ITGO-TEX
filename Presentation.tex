\begin{frontmatter}

\title{A Constrained Iterative Topographical Global \\ Optimization Applied to Engineering Optimization}

%\tnotetext[mytitlenote]{Fully documented templates are available in the elsarticle package on \href{http://www.ctan.org/tex-archive/macros/latex/contrib/elsarticle}{CTAN}.}


%% Group authors per affiliation:
%\author{Elsevier\fnref{footUFT}}
%\address{Radarweg 29, Amsterdam}
%\fntext[footUFT]{Universidade Federal do Tocantins}


\author[label1]{Matheus~Pedroza~Ferreira}
\ead{matheuspedrozaferreira@uft.edu.br}

\author[label1]{Marcelo~Lisboa~Rocha\corref{cor1}}
\ead{mlisboa@uft.edu.br}

\author[label2]{Ant\^onio~J.~Silva~Neto}
\ead{ajsneto@iprj.uerj.br}

\author[label3]{Wagner~F.~Sacco}
\ead{wagner.sacco@ufopa.edu.br}



\address[label1]{Departamento de Ci\^encia da Computa\c{c}\~ao, Universidade Federal do Tocantins, Quadra 109 Norte, Avenida NS-15, ALCNO-14, Palmas, Tocantins, Brazil}

\address[label2]{Departamento de Engenharia Mec\^anica e Energia, Instituto Polit\'ecnico, Universidade do Estado do Rio de Janeiro IPRJ/UERJ, RJ, Brazil}

\address[label3]{Instituto de Engenharia e Geociências, Universidade Federal do Oeste do Pará, PA, Brazil}



\cortext[cor1]{Corresponding authors}










\iffalse

\fntext[footMat]{Dept. de Ci\^encia da Computa\c{c}\~ao, Universidade Federal do Tocantins. \\ \textit{E-mail}: \textbf{matheuspedrozaferreira@uft.edu.br}}

\fntext[footLis]{Dept. de Ci\^encia da Computa\c{c}\~ao, Universidade Federal do Tocantins. \\ \textit{E-mail}: \textbf{mlisboa@uft.edu.br}}

\fntext[footAnt]{Departamento de Engenharia Mec\^anica e Energia, Instituto Politécnico, Universidade do Estado do Rio de Janeiro IPRJ/UERJ. \\ \textit{E-mail}: \textbf{ajsneto@iprj.uerj.br}}

\fi





%% or include affiliations in footnotes:
%\author[mymainaddress,mysecondaryaddress]{Elsevier Inc}
%\ead[url]{www.elsevier.com}

%\author[mysecondaryaddress]{Global Customer Service\corref{mycorrespondingauthor}}
%\cortext[mycorrespondingauthor]{Corresponding author}
%\ead{support@elsevier.com}



\begin{abstract}
    Nonlinear optimization is an active line of research, given the wide range of scientific fields that benefit from its development. In the last years, the meta-heuristics proved to be one of the most effective methods to tackle difficult optimization problems, providing an alternative in cases where exact methods would be unfeasible. In this work, we developed a method based on the Iterative Topographical Global Optimization meta-heuristic, which we call C-ITGO, incorporating specific mechanisms to solve nonlinearly constrained optimization problems. We use the method developed in this work to optimize eight complex engineering design problems and compare the results obtained here against several methods found in the literature. In the tests performed, C-ITGO outperforms all competing methods, achieving state of the art results for the problems considered.
\end{abstract}


\begin{keyword}
Meta-heuristics \sep ITGO \sep Optimization \sep Engineering Problems.
\end{keyword}


%\begin{keyword}
%Metaheuristics \sep BRKGA \sep Global Optimization \sep Local Search.
%\MSC[2010] 00-01\sep  99-00
%\end{keyword}


\end{frontmatter}


%\blfootnote{\fnt{9} \textit{Key words.}  Metaheuristics, BRKGA, Global Optimization, Local Search.}




%\linenumbers