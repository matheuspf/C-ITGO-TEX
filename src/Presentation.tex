%\begin{frontmatter}

\title{A Constrained ITGO Heuristic Applied to Engineering Optimization}


\author{Matheus~P.~Ferreira     \and
        Marcelo~L.~Rocha         \and
        Ant\^onio~J.~Silva~Neto      \and
        Wagner~F.~Sacco
}



\institute{M. P. Ferreira \and M. L. Rocha \at
           Departamento de Ci\^encia da Computa\c{c}\~ao, Universidade Federal do Tocantins, TO, Brazil \\
           \email{matheuspedrozaferreira@uft.edu.br, \\mlisboa@uft.edu.br} % 
           \and
           A. J. Silva Neto \at
           Departamento de Engenharia Mec\^anica e Energia, Instituto Polit\'ecnico, Universidade do Estado do Rio de Janeiro, IPRJ/UERJ, RJ, Brazil \\
           \email{ajsneto@iprj.uerj.br} %
           \and
           W. F. Sacco \at
           Instituto de Engenharia e Geociências, Universidade Federal do Oeste do Pará, PA, Brazil \\
           \email{wagner.sacco@ufopa.edu.br} %
}




\date{Received: date / Accepted: date}

\maketitle

\begin{abstract}
    Nonlinear optimization is an active line of research, given the wide range of scientific fields that benefit from its development. In the last years, the meta-heuristics proved to be one of the most effective methods to tackle difficult optimization problems, providing an alternative in cases where exact methods would be unfeasible. In this work, we present a method based on the Iterative Topographical Global Optimization meta~-heuristic, which we call C-ITGO, incorporating specific mechanisms to solve nonlinearly constrained optimization problems. We use the method developed in this work to optimize eight complex engineering design problems and compare the results obtained here against those obtained with several other methods found in the literature. In the tests performed, C-ITGO outperforms all competing methods, achieving state of the art results for the problems considered.

    \keywords{Meta-heuristics \and ITGO \and Optimization \and Engineering Problems.}
\end{abstract}



%\begin{keyword}
%Metaheuristics \sep BRKGA \sep Global Optimization \sep Local Search.
%\MSC[2010] 00-01\sep  99-00
%\end{keyword}


%\end{frontmatter}


%\blfootnote{\fnt{9} \textit{Key words.}  Metaheuristics, BRKGA, Global Optimization, Local Search.}




%\linenumbers